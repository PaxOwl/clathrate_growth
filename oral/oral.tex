\section{Introduction}
Découvertes en 1910 par Humphry Davy, les clathrates hydrates de methane sont des composés cristallins de molécules d'eau se présentant sous forme de cage et emprisonnant des molécules de méthane. Initiallement, elles étaient uniquement considérées comme des "curiosités de laboratoire", cependant Hammerschmidt a découvert en 1934 que les clathrates pouvaient se former dans les pipeline (oléoducs) créant des bouchons qui peuvent endommager le matériel, voir le détruire. C'est pourquoi depuis les années 50, de plus en plus de recherches sont menées à ce sujet, et elles sont encore très actives de nos jours.

L'étude réalisée pendant cette année porte sur la croissance et la fonte des clathrates. 

\section{Simulation de croissance et de fonte}

